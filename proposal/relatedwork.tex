\section{Related Work}

Major approaches to the word sense disambiguation problem can be briefly
categorized into three types: Knowledge-based, supervised learning, and
unsupervised learning. 

Knowledge based appraoch.
Lesk's algorithm~\cite{lesk1986automatic} is a
dictionary-based approach. It assumes that words appear within the same context
have related semantics. The alogrithm disambiguate word sense by finding the
dictionary definition that shares the most common words with the given context
sentence.

Supervised learning approaches. These approaches typically use a hand-annotated
Corpus as the training data set. The classifier is concerned on each individual
word, and the output is one of the definition of the ambiguous word in the
dictionary. Over the decades, people have proposed different algorithms to solve
this classification problem: decision-tree~\cite{quinlan1986induction}, 
Naive Bayers classifier~\cite{ng1997getting}, Neural
Network~\cite{cottrell1985connectionist}, and SVM~\cite{lee2002empirical}, etc.

Unsupervised approach. Such approaches~\cite{schutze1998automatic} assumes that
similar senses appear in similar contexts. By clustering occurrance of words
into clusters using some similarity measures, the sense of a new occurring word
can be induced by assigning it to the closest cluster.

% \textbf{Context representation.} To feed sentences into our system, we need to
% encode the context into some data structure. Previous researches have proposed
% several ways to represent context.
